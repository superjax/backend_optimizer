% !TEX root=main.tex

\section{Implementation}
There are several important details which are critical to demonstrating an effective relative edge-based optimization in hardware.  We will now discuss these details as they pertain to implementation on a small multirotor robot.

% talk briefly about front-end implementation
% talk about how edges are created
% talk about simulation simplification, define what an edge means and why we are only doing SE2
% Talk about specific algorithms being used
\subsection{Front-end State Estimation}Front-end state estimation must take place in real-time onboard a MAV. This state estimator must also provide mutually independent edge constraints to the back-end optimization routine.  In our implementation, we leverage the Relative Multiplicative Extended Kalman Filter (RMEKF)~\cite{Koch2017}, which has been demonstrated as an accurate filter-based front-end estimator which also provides mutually independent edge constraints to build a global pose graph.

As noted in~\cite{Wheeler2017a}, when operating a MAV over a flat ground plane, only the global position and heading states are unobservable to an agent equipped with relative sensors, an altimeter and an IMU.  Therefore, we only optimize over the unobservable states, which reduces the back-end optimization problem to transforms in $SE(2)$.  Therefore, we define edges as homogeneous transforms in $SE(2)$ as described in section~\ref{sec:relative_edge_optimization_in_se2}.

\subsection{Loop Closure Discovery and Calculation}
In addition to odometry constraints from the front-end estimation routine, we also require loop closure constraints when a vehicle observes landmarks it or another vehicle has viewed previously.  In our implementation, we utilize Fast, Appearance-Based Mapping (FAB-MAP) as our place-recognition algorithm and RGBD visual odometry techniques~\cite{Leishman2013} to calculate the full 6-Degree-Of-Freedom loop closure constraint between appearance-based matches.  These constraints have the same form as an odometry edge, and are also independent of other edges.  Therefore there can be considered homogeneously.

% Talk about what data is being shared
\subsection{Data Sharing and Communication Network}
