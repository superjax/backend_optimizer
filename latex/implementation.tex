% !TEX root=main.tex

\section{Implementation}
As alluded to before, there are several important details which are critical to an effetive relative edge-based optimization.  We will now discuss these details as they pertain to implementation in hardware on a MAV.

% talk briefly about front-end implementation
\subsection{Front-end State Estimation}
Front-end state estimation must take place in real-time onboard a MAV. This state estimator must also provide mutually independent edge constraints to the back-end optimization routine.  In our implementation, we leverage the Relative Multiplicative Extended Kalman Filter (RMEKF)~\cite{Koch2017}, which has been demonstrated as an accurate filter-based front-end estimator which also provides mutually independent edge constraints to build a global pose graph.

\subsection{Loop Closure Discovery and Calculation}
In addition to odometry constraints from the front-end estimation routine, we also require loop closure constraints when a vehicle observes landmarks it or another vehicle has viewed previously.  In our implementation, we utilize Fast, Appearance-Based Mapping (FAB-MAP) as our place-recognition algorithm and RGBD visual odometry techniques~\cite{Leishman2013} to calculate the full 6-Degree-Of-Freedom loop closure constraint between appearance-based matches.  These constraints have the same form as an odometry edge, are also independent of other edges and can therefore be considered homogeneously.

\subsection{Stochastic Selection of Edges}


% talk about simulation simplification, define what an edge means and why we are only doing SE2
% talk about how edges are created
% Talk about what data is being shared
% Talk about specific algorithms being used
